
%-------------------------------------------------------------------------------
% WORK EXPERIENCE
%-------------------------------------------------------------------------------

% Section header
\noindent\spacedlowsmallcaps{Work Experience}\vspace{1em}

\NewEntry{2014--Present}{Mathematical Modeling at TOTAL.} %\textsc{TOTAL E\&P USA} --- Houston}

\Description{\MarginText{TOTAL E\&P USA}Developed spreadsheets for risk analysis on exotic derivatives on a wide array of commodities (ags, oils, precious and base metals), managed blotter and secondary trades on structured notes, liaised with Middle Office, Sales and Structuring for bookkeeping. \\ Reference: John \textsc{McDonald}\ \ $\cdotp$\ \ +1 (000) 111 1111\ \ $\cdotp$\ \ \href{mailto:john@lehman.com}{john@lehman.com}}

%------------------------------------------------


\newcommand{\emailDavidHolman}{david.holman@nextlimit.com}
\NewEntry{2013--2014}{Mathematical Software Developer.}%\textsc{Xflow-CFD}  --- Madrid, Spain}

\Description{\MarginText{Xflow-CFD}XFLow is a high end, high efficiency CFD, Computational Fluid Dynamics, software.\\
\begin{itemize}
\item[-] Design and developed a new data architecture, this architecture must be very efficient, lightweight and easy to maintain and modify.
\item[-] Part of the data structure designed to be used as graphic user interface was developed with Qt library. 
\item[-] I solve several issues related to XFLow graphic user interface (GUI).
\end{itemize}

 Reference: David \textsc{Holman}\ \ $\cdotp$\ \ \href{mailto:\emailDavidHolman}{\emailDavidHolman}}

%------------------------------------------------

\NewEntry{Jun-Nov 2011}{Software Developer.}

\Description{\MarginText{Tb-Solutions}
\begin{itemize}
\item[-] I Developed an Eclipse plug-in which was used to easily develop the new MC-Server, that is an application for online banking.
\item[-] I took part on the development of the application user interface, of MC-Server.
\end{itemize}}

%------------------------------------------------

\newcommand{\webSiteESEO}{http://www.esa.int/Education/ESEO_mission}
\newcommand{\webGME}{gme.unizar.es}

\NewEntry{ 2009--2010}{Fellowship in Space Mechanical Group.}

\Description{\MarginText{ESA student project}ESEO's is a micro-satellite, who was completely developed by European Students as part of an ESA ( European Space Agency ) program.
\begin{itemize}
\item[-] I took part inside Mission Analysis (MIAS) team. We develop a code implementation to accurately calculate satellite orbit insertion and the subsequent orbit to all the mission. 
\item[-] Calculate satellite orientation to the whole mission period, in order to calculate sun exposure.
\end{itemize}
{\href{\webSiteESEO}{ESEO Project.}}
\\ Reference: Julia \textsc{Mar\'in Yaseli de la Parra}\ \ $\cdotp$\ \ \href{mailto:julia.astrofisica@gmail.com}{julia.astrofisica@gmail.com}}

%------------------------------------------------

\vspace{1em} % Extra space between major sections
