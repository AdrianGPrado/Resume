
%-------------------------------------------------------------------------------
% EDUCATION
%-------------------------------------------------------------------------------

\spacedlowsmallcaps{Education}\vspace{1em}

\NewEntry{2011–2012}{Universidad de Zaragoza \& UPV/EHU.}

\Description{\MarginText{Comutational Mathematics (M.S.)}\ \ $\cdotp$\ \ \textit{Mathematical Modeling and Numerical Methods in Physics and Engineering.}\newline 
Thesis: \textit{CUDA implementation of integration rules within an hp-Finite Element code}\newline
Description: I enrolled in this Master to obtain a deep formation on applied mathematics. Inside this Master I learned different skills, such as data mining, applied statistics, and optimization. Nevertheless along this Master, I have obtained a deep formation in Finite Element Modeling. Under David Pardo and Ricardo Celorrio's direction, I have developed an implementation of a FEM integration problem using CUDA's technology. With this work, called “CUDA implementation of integration rules within an hp-Finite Element code", I acquired the necessary skills to develop FEM simulation software. It is specially remarkable, that to develop FEM software, I first have to learn how different FEM software pieces works, and learn who to use FEM software like Open Foam or CalculiX.
The master course it is oriented form an applied mathematical point of view, and it is supported by several enterprises where the more outlined are DEIMOS-SPACE, and GAMESA.
European Official Master, 60 ects credits.\newline
Advisors: Prof.~David \textsc{Pardo}  \ \ $\&$ \ \  Prof.~Ricardo \textsc{Celorrio}\newline
Contact: \href{mailto:\emailDavidPardo}{\emailDavidPardo} \ \ $\cdotp$\ \ \href{mailto:\emailRicardoCelorrio}{\emailRicardoCelorrio}}

%------------------------------------------------

\NewEntry{2007-2010}{Universidad de Zaragoza.}

\Description{\MarginText{Physics (B.S.)}\textbf{Commerce Specialization}\ \ $\cdotp$\ \ School: Science Faculty Zaragoza University.\newline
Description: This degree focussed heavily on important things such as personnel management and mundane paperwork.}

%------------------------------------------------

\vspace{1em} % Extra space between major sections

% Universidad de Zaragoza
% Master of Science (M.S.), Computational and Applied Mathematics, Nowadays
% 2011 – 2012
% I enrolled in this Master to obtain a deep formation on applied mathematics. Inside this Master I learned different skills, such as data mining, applied statistics, and optimization. Nevertheless along this Master, I have obtained a deep formation in Finite Element Modeling. Under David Pardo and Ricardo Celorrio's direction, I have developed an implementation of a FEM integration problem using CUDA's technology. With this work, called “CUDA implementation of integration rules within an hp-Finite Element code", I acquired the necessary skills to develop FEM simulation software. It is specially remarkable, that to develop FEM software, I first have to learn how different FEM software pieces works, and learn who to use FEM software like Open Foam or CalculiX. 

% The master course it is oriented form an applied mathematical point of view, and it is supported by several enterprises where the more outlined are DEIMOS-SPACE, and GAMESA.

% European Official Master, 60 ects credits.
% Activities and Societies: LEEM (http://leem.es/en/index.php), APSIDE


% Universidad de Zaragoza
% Universidad de Zaragoza
% Bachelor of Science (B.S. 5 year Bachelor), Physics
% 2001 – 2010
% I learnt to simulate physical systems in C & C++, including Monte-Carlo simulation methods, and FEM simulation methods, differential equation mathematical models, non linear dynamical equations. Particle physics, nuclear physics, radiation protection and dosimetry, statistical behavior of atoms and molecules, thermodynamic, magnetism, laser, optics, solid state physics, quantum mechanics.

% I learnt to: (a) study and analyze different physical phenomena, and the corresponding governing laws, (b); to experiment with the properties of matter and sources or energy, (c); develop IT and industrial applications.


% Other qualifications:
% - Applied course on Advanced Computation (Jan 2010, Univ. Zaragoza).
% - Course on Renewable Energies (Mar 2010, Univ. Zaragoza).
% Activities and Societies: LEEM, APSIDE, Member of the students representatives from 2001 to 2003
% Telecom Satin-Etienne (http://www.telecom-st-etienne.fr/)
% Erasmus Exchage Programme, Telecommunications Engineering
% 2007 – 2008
% During my stay in the program Erasmus, lowers the guardianship of Jorat Luc and Virginie Fresse, my formation was the correspondent to a course in the Telecom Saint-Etienne, former ISTASE: guide of waves, antennas, optical fibers.
